% !TeX root = ../thuthesis-example.tex

\begin{denotation}[3cm]
  \item[GDPR] 欧盟于2018年5月25日正式施行的《通用数据保护条例》
  \item[CCPA] 美国《加州消费者隐私法案(California Consumer Privacy Act)》 
  \item[PCA] Principal components analysis,主成分分析法
  \item[SVM] Support Vector Machine,支持向量机
  \item[CNN] Convolutional Neural Network,卷积神经网络
  \item[LFW] Labled Faces in the Wild,人脸数据集
  \item[API] Application Programming Interface,应用程序接口
  \item[k-means] 一种计算机聚类算法
  \item[Fine-Tuning] 一种神经网络训练方法,俗称微调
  \item[ReLU] 神经网络的一种非线性激活函数
  \item[Sigmoid] 神经网络的一种非线性激活函数
  \item[Affine] 仿射转换,对向量做空间平移的操作
  \item[Softmax] 一种数学计算公式,将有限项离散概率分布的梯度对数归一化
  \item[MNIST] Mixed National Institute of Standards and Technology database,是美国国家标准与技术研究院收集整理的大型手写数字数据库
  \item[CIFAR-10] 是一个用于识别普适物体的小型数据集,有10个类别,每个类别有6000个图像
  \item[LEARNING\_RATE] 学习率超参数
  \item[WEIGHT\_DECAY] 权重衰减超参数
  \item[MOMENTUM] 训练冲量超参数  
  \item[BATCH\_SIZE] 每次训练迭代使用的训练数据的个数
  \item[SGD]   Stochastic gradient descent,随机梯度下降训练方法
  \item[padding] 卷积运算前的边界填充
  \item[CPU] Central Processing Unit,中央处理单元
  \item[SSD] Solid-State Drive,固态驱动器
  \item[Pytorch] 是一个开源的Python机器学习库
  \item[ResNet18] 是一个18层的卷积神经网络结构
  \item[Epoch] 是神经网络训练过程需要用到的名词,1个epoch表示训练过了1遍训练集中的所有样本
  \item[Loss] 训练神经网络时损失函数的数值
  \item[$I_{acc}$] 本文指神经网络准确率指标
  \item[$v_{retain\_acc}$] 本文指使用保留集测得的网络准确率的数值
  \item[$v_{forget\_acc}$] 本文指使用遗忘集测得的网络准确率的数值
  \item[遗忘集] 训练数据中标签是遗忘类别的训练集 
  \item[保留集]  训练数据中标签不是遗忘类别的训练集
\end{denotation}



% 也可以使用 nomencl 宏包,需要在导言区
% \usepackage{nomencl}
% \makenomenclature

% 在这里输出符号说明
% \printnomenclature[3cm]

% 在正文中的任意为都可以标题
% \nomenclature{PI}{聚酰亚胺}
% \nomenclature{MPI}{聚酰亚胺模型化合物,N-苯基邻苯酰亚胺}
% \nomenclature{PBI}{聚苯并咪唑}
% \nomenclature{MPBI}{聚苯并咪唑模型化合物,N-苯基苯并咪唑}
% \nomenclature{PY}{聚吡咙}
% \nomenclature{PMDA-BDA}{均苯四酸二酐与联苯四胺合成的聚吡咙薄膜}
% \nomenclature{MPY}{聚吡咙模型化合物}
% \nomenclature{As-PPT}{聚苯基不对称三嗪}
% \nomenclature{MAsPPT}{聚苯基不对称三嗪单模型化合物,3,5,6-三苯基-1,2,4-三嗪}
% \nomenclature{DMAsPPT}{聚苯基不对称三嗪双模型化合物(水解实验模型化合物)}
% \nomenclature{S-PPT}{聚苯基对称三嗪}
% \nomenclature{MSPPT}{聚苯基对称三嗪模型化合物,2,4,6-三苯基-1,3,5-三嗪}
% \nomenclature{PPQ}{聚苯基喹噁啉}
% \nomenclature{MPPQ}{聚苯基喹噁啉模型化合物,3,4-二苯基苯并二嗪}
% \nomenclature{HMPI}{聚酰亚胺模型化合物的质子化产物}
% \nomenclature{HMPY}{聚吡咙模型化合物的质子化产物}
% \nomenclature{HMPBI}{聚苯并咪唑模型化合物的质子化产物}
% \nomenclature{HMAsPPT}{聚苯基不对称三嗪模型化合物的质子化产物}
% \nomenclature{HMSPPT}{聚苯基对称三嗪模型化合物的质子化产物}
% \nomenclature{HMPPQ}{聚苯基喹噁啉模型化合物的质子化产物}
% \nomenclature{PDT}{热分解温度}
% \nomenclature{HPLC}{高效液相色谱(High Performance Liquid Chromatography)}
% \nomenclature{HPCE}{高效毛细管电泳色谱(High Performance Capillary lectrophoresis)}
% \nomenclature{LC-MS}{液相色谱-质谱联用(Liquid chromatography-Mass Spectrum)}
% \nomenclature{TIC}{总离子浓度(Total Ion Content)}
% \nomenclature{\textit{ab initio}}{基于第一原理的量子化学计算方法,常称从头算法}
% \nomenclature{DFT}{密度泛函理论(Density Functional Theory)}
% \nomenclature{$E_a$}{化学反应的活化能(Activation Energy)}
% \nomenclature{ZPE}{零点振动能(Zero Vibration Energy)}
% \nomenclature{PES}{势能面(Potential Energy Surface)}
% \nomenclature{TS}{过渡态(Transition State)}
% \nomenclature{TST}{过渡态理论(Transition State Theory)}
% \nomenclature{$\increment G^\neq$}{活化自由能(Activation Free Energy)}
% \nomenclature{$\kappa$}{传输系数(Transmission Coefficient)}
% \nomenclature{IRC}{内禀反应坐标(Intrinsic Reaction Coordinates)}
% \nomenclature{$\nu_i$}{虚频(Imaginary Frequency)}
% \nomenclature{ONIOM}{分层算法(Our own N-layered Integrated molecular Orbital and molecular Mechanics)}
% \nomenclature{SCF}{自洽场(Self-Consistent Field)}
% \nomenclature{SCRF}{自洽反应场(Self-Consistent Reaction Field)}
