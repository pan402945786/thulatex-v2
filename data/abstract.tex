% !TeX root = ../thuthesis-example.tex

% 中英文摘要和关键字

\begin{abstract}
  随着隐私保护越来越受到重视,一些利用用户隐私数据学习的机器学习模型需要具有遗忘功能。
  卷积神经网络被应用到生活的许多方面,可是基于卷积神经网络的遗忘还没有得到足够的重视。
  本文利用了卷积神经网络的分层抽象特性设计了一套遗忘方法,能够使基于卷积神经网络训练的模型达到足够好的遗忘效果,同时不影响未被遗忘类别的准确率。
  除此之外,我们还引用了三个评价遗忘效果的指标。
  我们对方法达到的遗忘效果进行了实验验证,同时与完全重新训练方法进行了对比。
  实验结果表明,本文提出的方法达到了理想的遗忘效果,并且随着遗忘类别的增加也不会使遗忘效果变坏。

  在文章最后对本文的不足进行了分析,并且对基于卷积神经网络的遗忘方法的发展进行了展望。

  % 关键词用“英文逗号”分隔,输出时会自动处理为正确的分隔符
  \thusetup{
    keywords = {隐私保护, 遗忘, 卷积神经网络},
  }
\end{abstract}

\begin{abstract*}
  As privacy protection is paid more and more attention, some machine learning models that use user privacy data for learning need to have forgetting functions.
  Convolutional neural networks are applied to many aspects of life, but forgetting based on convolutional neural networks has not received enough attention.
  This paper uses the hierarchical abstract characteristics of the convolutional neural network to design a set of forgetting methods, which can make the model based on the convolutional neural network training achieve a good enough forgetting effect, while not affecting the accuracy of the unforgettable category.
  In addition, we also cited three indicators to evaluate the effect of forgetting.
  We experimented to verify the forgetting effect achieved by the method, and compared it with the complete retraining method.
  The experimental results show that the method proposed in this paper achieves the ideal forgetting effect, and will not make the forgetting effect worse as the forgetting category increases.

  At the end of the article, the deficiencies of this article are analyzed, and the development of forgetting methods based on convolutional neural networks is prospected.
  % Use comma as seperator when inputting
  \thusetup{
    keywords* = {Privacy Preserving, Forgetting, Convolutional Neural Network},
  }
\end{abstract*}
